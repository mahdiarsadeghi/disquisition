\documentclass[10pt]{beamer}

\mode<presentation> {
\usetheme{Goettingen}
}

\usepackage[english]{babel}
\usepackage[latin1]{inputenc}
\usepackage[T1]{fontenc}

\usepackage{graphicx} 
\usepackage{tikz}
\usepackage{booktabs} 
\usepackage{amsmath}
\usepackage{url,color}
\usepackage{subfigure}
\usepackage{graphicx}
\usepackage{xcolor}
\usepackage{listings,,lstautogobble}
\usepackage{amsthm,amsfonts,amssymb,amscd,amsxtra}
\lstset{language=Java,
	keywordstyle=\color{RoyalBlue},
	basicstyle=\scriptsize\ttfamily,
	commentstyle=\ttfamily\itshape\color{gray},
	stringstyle=\ttfamily,
	showstringspaces=false,
	breaklines=true,
	frameround=ffff,
	frame=single,
	rulecolor=\color{black},
	autogobble=true
}
\usepackage{biblatex}
\addbibresource{references.bib}

\title[Identifiability Tutorial]{Computational Identifiability Analysis Tutorial} 

\author{Mahdiar Sadeghi} 
\medskip

\date{November 5, 2021} 

\begin{document}

\begin{frame}
\titlepage % Print the title page as the first slide
\end{frame}

%-------------------------------------------------%
\section{Outline}

\begin{frame}{Computational identifiability analysis}
	The objective is to review the mathematical basics of identifiability and a tutorial to computational packages available. 
	\vspace{15pt}
    \begin{itemize}
        \setlength\itemsep{1em}
        \item \textbf{Review:} mathematical definitions for identifiability.
        
        \item \textbf{SIAN:} an open source software for structural identifiabilty.
    
    	\item \textbf{Profile likelihood:} a useful tool for practical identifiability.
    \end{itemize}
\end{frame}

%-------------------------------------------------%
\section{Review}

\begin{frame}{Review of basic definitions.}
	The basic definitions are from Miao et. al.~\footfullcite{miao2011identifiability}, for a general system:
	
	\begin{subequations}
		\begin{align}
			\dot x (t) &= f(t,x(t),u(t),\theta), \\
			y(t) &= g(x(t), u(t), \theta).
		\end{align}
	\end{subequations}
	
	\begin{itemize}
		\item $x(t) \in R^n$ is a vector of state variables.
		\item $y(t) \in R^m$ is the measurement or output vector.
		\item $u(t) \in R^p$ is the known system input vector.
		\item $\theta \in R^q$ the parameter vector.
		\item $\theta$ is unknown and has to be estimated based on experimental data. Here we assume that the parameters are constant.
	\end{itemize}
\end{frame}

\subsection{Global identifiability}

\begin{frame}{Global identifiabilty}
	\begin{itemize}
		\item \textbf{Definition: } A system structure is said to be globally identifiable if for any $\theta$ within an open neighborhood of some point $\theta^*$ in the parameter space, $y(u,\theta_1)=y(u,\theta_2)$ holds if and only if $\theta_1=\theta_2$.
		\item \textbf{Example: } The following model is globally identifiable.
			\begin{subequations} \label{eq:1}
				\begin{align}
					\dot x(t) &= a x(t) + b u(t), \\
					y(t) &= x(t).
				\end{align}
			\end{subequations}
		\item \textbf{Example: } The following model is not globally identifiable.
		\begin{subequations} \label{eq:2}
			\begin{align}
				\dot x(t) &= \frac{a}{c} x(t) + b u(t), \\
				y(t) &= x(t).
			\end{align}
		\end{subequations}
		$y(t)$ of parameter sets $\theta_1=(a,b,c)= (1,2,3)$ and $\theta_2=(a,b,c)= (2,2,6)$ are the same.
	\end{itemize}
	
\end{frame}

\subsection{More definitions}

\begin{frame}{More identifiabilty definitions}
	\begin{itemize}
		\item \textbf{Local identifiability: } A system structure is said to be locally identifiable if for any admissible input $u(t)$ and any two parameter vector $\theta_1$ and $\theta_2$ in the prameter space $\Theta$, $y(u,\theta_1) = y(u,\theta_2)$ holds if an only if $\theta_1=\theta_2$.
		\item \textbf{Local strong identifiability } ($x_0$-identifiability). 
		\item \textbf{Structural identifiability.} 
		\item \textbf{Algebric identifiability.} 
		\item \textbf{Algebric identifiability with known initial conditions.} 
		\item \textbf{Structural equivalence: } Given two systems in the form
		\begin{subequations}
			\begin{align}
				\dot x &= f(x(t,\theta)\theta) + u(t) g(x(t,\theta), \theta),\\
				y &= h(x(t,\theta),\theta)
			\end{align}
		\end{subequations}
		If there exist two parameters $\theta_1,\theta_2 \in \Theta$ such that, for the same admissible input $u(t)$, the solution of the two systems exists for $\theta_1$ and $\theta_2$, respectively, and the corresponding system outputs are the same, the system with parameters $\theta_1$ is said to be equivalent to the system with parameter $\theta_2$.
	\end{itemize}
	
\end{frame}


%-------------------------------------------------%
\section{SIAN}

\begin{frame}{Structural Identifiability Analyzer (SIAN)\footfullcite{hong2019sian}}
	\begin{itemize}
		\item Open source structural identifiability software in Maple.
		\item Here is a benchmark for comparing SIAN performance with other available identifiability pacakges.
			\begin{figure}
				\includegraphics[width=1\linewidth]{sian-benchmark.png}
			\end{figure}
		\item A Maple app running SIAN online: \url{https://maple.cloud/app/6509768948056064}.
		\item A Julia package which often outperforms SIAN: \url{https://github.com/SciML/StructuralIdentifiability.jl}.
	\end{itemize}
\end{frame}


\subsection{Basic examples}

\begin{frame}[fragile]{Copy the following examples into Maple cloud app.}
	Model \eqref{eq:1} where the parameters and initial conditions should be locally and globally identifiable:
	\begin{lstlisting}
		dx/dt = a*x + b*u(t)
		y=x
	\end{lstlisting}
\vspace{20pt}
	Model \eqref{eq:2} where $a$, and $c$ are not identifiable:
	\begin{lstlisting}
		dx/dt = a*x/c + b*u(t),
		y=x
	\end{lstlisting}
\end{frame}

\subsection{Tumor growth}

\begin{frame}[fragile]{Tumor growth model~\footfullcite{simeoni2004predictive}.}
	The minimal version of the tumor growth model with $n=2$:
	\begin{subequations}
		\begin{align}
			\dot x_1(t) &= \frac{\lambda_0 x_1(t)}{(1+(\frac{\lambda_0 y(t)}{\lambda_1})^\Psi)^\frac{1}{\Psi}} - k_2 x_1(t) u(t), \\
			\dot x_2(t) &= k_2 x_1(t) u(t) - k_1 x_2(t), \\
			y(t) &= x_1(t) + x_2(t).
		\end{align}
	\end{subequations}
	Let's try SIAN with the following code.
	\begin{lstlisting}
		dx1/dt = l0*x1/(1+(l0*y/l1)^P)^(1/P) - k2*x1*u(t),
		dx2/dt = k2*x1*u(t) - k1*x2,
		y = x1+x2
	\end{lstlisting}
	\color{red} error message:StringTools:-RegMatch. \\ 
	\color{black} The solution is to use an equivalence form of the model. But why?
	\begin{lstlisting}
		dx1/dt = x1*w-k2*u(t)*x1, 
		dx2/dt = k2*u(t)*x1-k1*x2,
		dz/dt = psi*(x1*w-k1*x2)*z/y,
		dw/dt = -(x1*w-k1*x2)*z*w/(y*(1+z)),
		y= x1+x2
	\end{lstlisting}
\end{frame}

\subsection{Parameter as power}

\begin{frame}{In general, the algorithm does not allow noninteger powers and powers being parameters~\footnote{https://github.com/pogudingleb/SIAN/issues/2}}
	
	The solution is to define intermediate variables $z(t)$ and $w(t)$:
	\begin{subequations}
		\begin{align}
			z(t) &= (\frac{\lambda_0 y(t)}{\lambda_1})^\Psi, \\
			w(t) &= \frac{\lambda_0}{(1+z(t))^\frac{1}{\Psi}}.
		\end{align}
	\end{subequations}
	Then dynamic can be rewritten in the following form.
	\begin{subequations}
		\begin{align}
			\dot x_1(t) &= x_1(t) w(t) - k_2 u(t) x_1(t), \\
			\dot x_2(t) &= k_2 u(t) x_1(t) - k_1 x_2(t), \\
			\dot z(t) &=  \frac{\Psi z(t)}{y(t)} \dot y(t) = 
			\frac{\Psi}{y(t)} (x_1(t) w(t) - k_1 x_2(t)), \\
			\dot w(t) &= - \frac{w(t) \dot z(t)}{\Psi (1+z(t))} = 
			- \frac{w(t) z(t)}{y(t) (1+z(t))} (x_1(t) w(t) - k_1 x_2(t)).
		\end{align}
	\end{subequations}
\end{frame}

%-------------------------------------------------%
\section{Profile likelihood}

\begin{frame}{Profile likelihood}
	\begin{itemize}
		\item Most useful to determine practical identifiability based on a given synthetic/expermental data.
		\item Practical identifiability $\Rightarrow$ structural identifiabiility.
		\item Structural identifiablity $\nRightarrow$ practical identifiablity.
		\vspace{10pt}
		\item An open source software in MATLAB~\footfullcite{raue2015data2dynamics}: 
		\url{https://github.com/Data2Dynamics/d2d}.
		\item An open source software in Python~\footfullcite{parra2020pdeparams}: \url{https://github.com/systemsmedicine/PDE_params}.
	\end{itemize}
\end{frame}

\subsection{Basic examples}

\begin{frame}
	content...
\end{frame}

\end{document}