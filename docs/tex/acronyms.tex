\chapter*{List of Acronyms}
\addcontentsline{toc}{chapter}{List of Acronyms}

% below is the list of acronym definitions, place them in alphabetical order
% since they will not be sorted again. 
\begin{acronym}
	\acro{AUC}{Area Under the Curve}.
	The definite integral of a curve that describes the variation of a drug concentration in blood plasma as a function of time.
	
	\acro{FDA}{Food and Drug Administration}.
	The United States Food and Drug Administration is a federal agency of the Department of Health and Human Services.
	
	\acro{MDOR}{Mathematically Derived Optimal Regimen}.
	
	\acro{MTD}{Maximum Tolerated Dose}.
	The maximum tolerated dose is commonly estimated to be the maximum dose that can be administered for the duration of a specific study that will not compromise the survival of the animals by causes other than carcinogenicity.
	
	\acro{NPI}{Nonpharmaceutical Intervention}.
	Actions, apart from getting vaccinated and taking medicine, that people and communities can take to help slow the spread of illnesses like pandemic influenza (flu).
	
	\acro{QSS}{Quasi Steady State}.
	A situation that is changing slowly enough that it can be considered to be constant.
	
	\acro{SCID}{Severe combined immunodeficiency}
	SCID mice have a genetic immune deficiency that affects their B and T cells. Due to the lack of mature B and T lymphocytes, these mouse models are ideal for xenoengraftment of human cells and tissue.
	
	\acro{SD}{Social Distancing}.
	In public health, social distancing, also called physical distancing, is a set of non-pharmaceutical interventions or measures intended to prevent the spread of a contagious disease by maintaining a physical distance between people and reducing the number of times people come into close contact with each other.
	
	
\end{acronym}
