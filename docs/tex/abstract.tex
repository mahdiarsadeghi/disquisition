% abstract.tex:

\begin{abstract}

% basic intro
Theory and practice are the two fundamental tools in engineering and scientific research. 
With a great increase of quantitative experiments in biological systems over the past decades, mathematical modeling is able to enhance predictions and generate new hypotheses.
% background
A ``good" model of a system, that is expected to reproduce the experimental observations, is capable of making predictions outside the previous experimental settings. 
However, the accuracy of predictions based on mathematical models highly depends on the assumptions used to model the system.
% general problem
The objective of this study is to explore possible approaches to deploy such models in order to find new hypotheses to be tested in future experimental settings.
% how we did it
From the lens of control and decision-making, a few biological systems relevant to chemotherapy, immunotherapy, and epidemics are considered in this work. Models are analyzed numerically and analytically in order to enhance the outcome of the system with a new control/decision.
% main result
A new dosing plan for chemotherapy is identified and evaluated via in-silico experiments to optimally reduce the tumor volume at the end of the plan. The new dosing plan consists of two doses starting with a small dose at the beginning of the plan and an increased dose after a few weeks. Unlike traditional chemotherapy plans currently used, the proposed plan is neither a maximum tolerated dose, nor a metronomic/intermittent plan. 
% general context
Moreover, epidemic models under social distancing guidelines are studied. Considering a single interval social distancing based on the start time and the duration of the social distancing shows a linear relationship between optimal timing of the social distancing.
% broader perspective 
Models analyzed in this work are generic and applicable to wide a range of applications. 

\end{abstract}


%Theory and practice are the two fundamental tools in engineering and scientific research. With a great increase of quantitative experiments in biological systems over the past decades, mathematical modeling is able to enhance predictions and generate new hypotheses. A ``good" model of a system, that is expected to reproduce the experimental observations, is capable of making predictions outside the previous experimental settings.  However, the accuracy of predictions based on mathematical models highly depends on the assumptions used to model the system. The objective of this study is to explore possible approaches to deploy such models in order to find new hypotheses to be tested in future experimental settings. From the lens of control and decision-making, a few biological systems relevant to chemotherapy, immunotherapy, and epidemics are considered in this work. Models are analyzed numerically and analytically in order to enhance the outcome of the system with a new control/decision. A new dosing plan for chemotherapy is identified and evaluated via in-silico experiments to optimally reduce the tumor volume at the end of the plan. The new dosing plan consists of two doses starting with a small dose at the beginning of the plan and an increased dose after a few weeks. Unlike traditional chemotherapy plans currently used, the proposed plan is neither a maximum tolerated dose, nor a metronomic/intermittent plan. Moreover, epidemic models under social distancing guidelines are studied. Considering a single interval social distancing based on the start time and the duration of the social distancing shows a linear relationship between optimal timing of the social distancing. Models analyzed in this work are generic and applicable to a wide range of applications. 